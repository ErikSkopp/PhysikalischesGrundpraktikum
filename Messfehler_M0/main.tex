\documentclass[fleqn,10pt]{olplainarticle}
% Use option lineno for line numbers 

\title{Messfehler - M0}

\author[1]{Skopp, Erik}

\author[2]{Skopp, Erik (60XXX) und Treßin, Devin (72XXX)}


\keywords{Messabweichung, Fadenpendel, M0, Grundpraktikum Physik}

\begin{abstract}
In diesem Experiment messen wir die Periodendauer eines Fadenpendels, um die Erdbeschleunigung zu berechnen. Dazu machen wir 200 Messungen und werten sie statistisch aus. Wir lernen, wie man zufällige und systematische Fehler erkennt und die Ergebnisse in Histogrammen darstellt, um den wahren Wert der Periodendauer und die Standardabweichung abzuschätzen..
\end{abstract}

\begin{document}

\flushbottom
\maketitle
\thispagestyle{empty}

\section*{Grundlagen}

Thanks for using Overleaf to write your article. Your introduction goes here! Some examples of commonly used commands and features are listed below, to help you get started.

\section{Aufgabenstellung}
\begin{enumerate}
    \item Die Periodendauer eines Fadenpendels ist mehrmals zu messen. Die Häufigkeitsverteilung der Messwerte ist in Abhängigkeit von der Anzahl $n$ der Messungen in geeigneten Histogrammen darzustellen. 

    \item Für $n = 200 $ Messwerte sind Schätzwerte für die Standardabweichung $ \sigma $  und den wahren Wert $\mu $ der Messgröße anzugeben und mit grafisch ermittelten Ergebnissen zu vergleichen. 

    \item  Aus Periodendauer und Länge des Pendels ist die Schwerebeschleunigung der Erde einschließlich ihrer kombinierten Unsicherheit zu berechnen. 
\end{enumerate}

\section{Kontrollfragen}
 \begin{enumerate}
     \item Was versteht man unter einer Messabweichung? In welche beiden Kategorien werden sie unterteilt?

     \item Wie sieht die Dichtefunktion der Normalverteilung aus? Wie sieht das zugehörige Integral aus und was kann man daraus ablesen? 

     \item Was gibt die Standardabweichung an?  
 \end{enumerate}

\section{Begriffserklärungen}

\end{document}